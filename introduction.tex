\chapter*{Introduction}
\addcontentsline{toc}{chapter}{Introduction}
\markboth{Introduction}{Introduction}
\label{chap:introduction}

 	Aujourd'hui, grâce à la grande évolution du domaines réseaux et embarqués , l'ensemble de l'infrastructure physique est étroitement lié aux technologies de l’information et de la communication, où la surveillance et la gestion intelligentes peuvent être réalisées grâce à l'utilisation de dispositifs embarqués en réseau : l'Internet des objets (IoT).\\ [1cm]
 	
 	L'IOT permet la mise en œuvre d'une plate-forme (mobile ou web) capable de surveiller via Internet
un ensemble des lectures environnementales avec l'utilisation de dispositifs de faible puissance appelés réseau de capteurs sans fil (RCSF).\\ [1cm]
    
    
    Les domaines d’applications des réseaux de capteurs sans fil sont de plus en plus élargi  telles que le domaine de la médecine, le domaine militaire, le domaine environnemental, le secteur industriel... Cela est dû au fonctionnalités pratiques des capteurs et l’élargissement des gammes de capteurs disponibles (mouvement, température, physiologique...)
et l’évolution des technologies des communications sans fil (Bluetooth et Zigbee) .

